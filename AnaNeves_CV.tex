%!TEX TS-program = xelatex
%!TEX encoding = UTF-8 Unicode
% Awesome CV LaTeX Template for CV/Resume
%
% This template has been downloaded from:
% https://github.com/posquit0/Awesome-CV
%
% Author:
% Claud D. Park <posquit0.bj@gmail.com>
% http://www.posquit0.com
%
%
% Adapted to be an Rmarkdown template by Mitchell O'Hara-Wild
% 23 November 2018
%
% Template license:
% CC BY-SA 4.0 (https://creativecommons.org/licenses/by-sa/4.0/)
%
%-------------------------------------------------------------------------------
% CONFIGURATIONS
%-------------------------------------------------------------------------------
% A4 paper size by default, use 'letterpaper' for US letter
\documentclass[11pt,a4paper,]{awesome-cv}

% Configure page margins with geometry
\usepackage{geometry}
\geometry{left=1.4cm, top=.8cm, right=1.4cm, bottom=1.8cm, footskip=.5cm}


% Specify the location of the included fonts
\fontdir[fonts/]

% Color for highlights
% Awesome Colors: awesome-emerald, awesome-skyblue, awesome-red, awesome-pink, awesome-orange
%                 awesome-nephritis, awesome-concrete, awesome-darknight

\definecolor{awesome}{HTML}{8e7cc3}

% Colors for text
% Uncomment if you would like to specify your own color
% \definecolor{darktext}{HTML}{414141}
% \definecolor{text}{HTML}{333333}
% \definecolor{graytext}{HTML}{5D5D5D}
% \definecolor{lighttext}{HTML}{999999}

% Set false if you don't want to highlight section with awesome color
\setbool{acvSectionColorHighlight}{true}

% If you would like to change the social information separator from a pipe (|) to something else
\renewcommand{\acvHeaderSocialSep}{\quad\textbar\quad}

\def\endfirstpage{\newpage}

%-------------------------------------------------------------------------------
%	PERSONAL INFORMATION
%	Comment any of the lines below if they are not required
%-------------------------------------------------------------------------------
% Available options: circle|rectangle,edge/noedge,left/right

\photo{./Images/profile.jpeg}
\name{Neves}{Ana}

\position{Psychology PhD Researcher}
\address{University of Sussex, Brighton, UK}

\email{\href{mailto:A.Neves@sussex.ac.uk}{\nolinkurl{A.Neves@sussex.ac.uk}}}
\github{AnafNeves}

% \gitlab{gitlab-id}
% \stackoverflow{SO-id}{SO-name}
% \skype{skype-id}
% \reddit{reddit-id}

\quote{\RaggedRight  Doctoral Tutor and PhD researcher in Psychology at
the University of Sussex, supervised by Dr Dominique Makowski. My
research explores the role of interoception - our perception of internal
bodily states - in shaping emotional experiences and beliefs about
reality. I use a combination of behavioural tasks and physiological
measures to investigate how bodily awareness influences responses to
emotionally evocative stimuli.}

\usepackage{booktabs}

\providecommand{\tightlist}{%
	\setlength{\itemsep}{0pt}\setlength{\parskip}{0pt}}

%------------------------------------------------------------------------------



% Pandoc CSL macros

\begin{document}

% Print the header with above personal informations
% Give optional argument to change alignment(C: center, L: left, R: right)
\makecvheader

% Print the footer with 3 arguments(<left>, <center>, <right>)
% Leave any of these blank if they are not needed
% 2019-02-14 Chris Umphlett - add flexibility to the document name in footer, rather than have it be static Curriculum Vitae
\makecvfooter
  {May 2025}
    {Neves Ana~~~·~~~Curriculum Vitae}
  {\thepage~ of \pageref{LastPage}~}


%-------------------------------------------------------------------------------
%	CV/RESUME CONTENT
%	Each section is imported separately, open each file in turn to modify content
%------------------------------------------------------------------------------



\normalsize

\section{Education}\label{education}

\begin{cventries}
    \cventry{University of Sussex}{Doctor of Philosophy (PhD) - Psychology}{UK}{2024 - ongoing}{\begin{cvitems}
\item supervised by \href{https://dominiquemakowski.github.io/}{\color[HTML]{1976D2}{Dr. Dominique Makowski}}
\end{cvitems}}
    \cventry{Univeristy of Sussex}{Research Masters in Psychological Methods}{UK}{2023 - 24}{}\vspace{-4.0mm}
    \cventry{University of Sussex}{Bachelor of Science (BSc) - Psychology with Cognitive Science}{UK}{2018 - 2022}{}\vspace{-4.0mm}
\end{cventries}

\section{Experience}\label{experience}

\subsection{Teaching}\label{teaching}

\begin{cventries}
    \cventry{University of Sussex}{Quantitative and Qualitative Methods}{Brighton, UK}{2025}{\begin{cvitems}
\item Facilitated students’ critical engagement with quantitative and qualitative research methods in psychology, with a particular focus on measurement issues and the interpretive decisions researchers make when investigating complex psychological constructs.
\end{cvitems}}
    \cventry{University of Sussex}{Discovering Statistics}{Brighton, UK}{2024}{\begin{cvitems}
\item Supported students in developing both theoretical understanding and practical skills in implementing, interpreting, and reporting statistical analyses, with a focus on the linear model.
\end{cvitems}}
\end{cventries}

\section{Skills}\label{skills}

\subsection{Language}\label{language}

\small

\begin{itemize}
\tightlist
\item
  \textbf{Portuguese}: Native
\item
  \textbf{English}: Fluent
\end{itemize}

\normalsize

\subsection{Research}\label{research}

\small

My PhD investigates how awareness that a stimulus is `fake' influences
emotional and belief-related responses, and seeks to identify the
factors that shape these reality judgments. A key focus is the role of
interoception in modulating affective responses - measured in terms of
intensity, valence, and eroticism. We are also keen on exploring the use
of transcutaneous vagus nerve stimulation (tVNS) to manipulate
interoceptive sensitivity, with the aim of determining whether such
modulation affects both physiological and subjective responses to
stimuli labelled as `fake', as well as their perceived realism.

\normalsize

\small

\subsection{Techniques, Methods and Open
Science}\label{techniques-methods-and-open-science}

I am currently developing technical expertise in EEG, physiological
signal processing, and computational methods including Bayesian
modelling. Aligned with my supervisor's
(\href{https://github.com/DominiqueMakowski}{\color[HTML]{1976D2}{\textbf{GitHub}}})
commitment to open science, I actively contribute to open-source
communities (e.g., on
\href{https://github.com/AnafNeves}{\color[HTML]{1976D2}{\textbf{GitHub}}})
to support accessibility and transparency in research.

I am committed to the highest standards of open and reproducible
science, routinely sharing data, materials, and analysis scripts. This
commitment is grounded in a strong emphasis on methodological rigour and
ethical responsibility throughout the research process.

\subsection{Other Projects}\label{other-projects}

I am also currently collaborating on a project outside my PhD with a PhD
researcher at Macquarie University in Sydney, Australia
(\href{https://github.com/Tam-Pham}{\color[HTML]{1976D2}{\textbf{GitHub}}}).
The project is a meta-analysis examining the association between
psychopathology and heart rate variability.

\section{Publications}\label{publications}

Makowski, D., Te, A. S., \textbf{Neves, A.}, Kirk, S., Liang, N. Z.,
Mavros, P., \& Chen, S. A. (2025).
\href{https://doi.org/10.1016/j.actpsy.2024.104670}{Too beautiful to be
fake: Attractive faces are less likely to be judged as artificially
generated}. \emph{Acta Psychologica}, 252, 104670.

\subsection{Pre-Prints}\label{pre-prints}

\small

Makowski, D., \textbf{Neves, A.}(2025).
\href{https://doi.org/10.31234/osf.io/873th_v3}{Testing the Relationship
between Phenomenological Control related to Illusion Sensitivity}

Makowski, D., Te, A. S., \textbf{Neves, A.}, \& Chen, S. A. (2024).
\href{https://doi.org/10.31234/osf.io/436np}{Measuring Depression and
Anxiety with 4 items? Adaptation of the PHQ-4 to increase its
Sensitivity to Subclinical Variability}

\normalsize

\section{References}\label{references}

\small

Contact in case of inquiry.

\begin{itemize}
\tightlist
\item
  \textbf{Dr Dominique Makowski} (PhD supervisor):
  \href{mailto:d.makowski@sussex.ac.uk}{\nolinkurl{d.makowski@sussex.ac.uk}}
\end{itemize}

\normalsize


\label{LastPage}~
\end{document}
